\documentclass[a4paper,11pt]{article}

\usepackage[a4paper,top=2cm,bottom=2cm,left=2cm,right=2cm]{geometry}
\usepackage[T1]{fontenc}
\usepackage[utf8]{inputenc}
\usepackage{graphicx}
\usepackage{lmodern}% http://ctan.org/pkg/lm

\graphicspath{{./Immagini/}} 

\newcommand{\tab}[1]{\hspace{.3\textwidth}\rlap{#1}}
\newcommand{\itab}[1]{\hspace{0em}\rlap{#1}}


\begin{document}

\begin{figure}[!htbp]
\centering	
	\mbox{%
		\begin{minipage}{.10\textwidth}
			\includegraphics[scale=0.20]{Immagini/unict.jpg} 
		\end{minipage}%
		\quad
		\begin{minipage}[c]{.60\textwidth}
			\centering
			\Large Universit\`{a} degli Studi di Catania
			
			\Large Dipartimento di Matematica e Informatica
			
			\Large Corso di Laurea in Informatica triennale
			
		\end{minipage}
	}
\end{figure}

\begin{center}
	\hrule
\end{center}

\vspace*{50pt}

\begin{center}
	\LARGE Andrea Costazza
\end{center}

\vspace*{30pt}

\begin{center}
	\LARGE \textbf{Titolo:}

	\LARGE \textbf{sottotitolo}
\end{center}

\vspace*{80pt}

\noindent\hfil\rule{0.2\textwidth}{.4pt}\hfil

\begin{center}
	\Large Relazione progetto finale
\end{center}

\noindent\hfil\rule{0.2\textwidth}{.4pt}\hfil

\vspace*{180pt}

\begin{flushright}
	\Large Relatore

	\Large \textbf {Prof. Domenico Cantone}
\end{flushright}

\bigskip
\bigskip

\hrule

\begin{center}
	\Large Anno Accademico 2015/16
\end{center}

\newpage

\begin{figure}[!htbp]
\centering	
	\mbox{%
		\begin{minipage}{.10\textwidth}
			\includegraphics[scale=0.20]{Immagini/unict.jpg} 
		\end{minipage}%
		\quad
		\begin{minipage}[c]{.60\textwidth}
			\centering
			\Large Universit\`{a} degli Studi di Catania
			
			\Large Dipartimento di Matematica e Informatica
			
			\Large Corso di Laurea in Informatica triennale
			
		\end{minipage}
	}
\end{figure}

\begin{center}
	\hrule
\end{center}

\vspace*{50pt}

\begin{center}
	\LARGE Andrea Costazza
\end{center}

\vspace*{30pt}

\begin{center}
	\LARGE \textbf{Titolo:}

	\LARGE \textbf{sottotitolo}
\end{center}

\vspace*{80pt}

\noindent\hfil\rule{0.2\textwidth}{.4pt}\hfil

\begin{center}
	\Large Relazione progetto finale
\end{center}

\noindent\hfil\rule{0.2\textwidth}{.4pt}\hfil

\vspace*{180pt}

\begin{flushright}
	\Large Relatore

	\Large \textbf {Prof. Domenico Cantone}
\end{flushright}

\bigskip
\bigskip

\hrule

\begin{center}
	\Large Anno Accademico 2015/16
\end{center}

\newpage
\null
\thispagestyle{empty}

\newpage

\LARGE{\textbf{Indice}}
\bigskip
  
\begin{enumerate}
	\item \LARGE{\textbf{Web Semantico}}
		\begin{enumerate}
			\Large
			\item Introduzione.
			\item Resource Description Framework (RDF).
			\item Logiche descrittive.
		\end{enumerate}
	\bigskip
	\item \LARGE{\textbf{Mappe on-line per siti web}}
		\begin{enumerate}
			\Large
			\item Leaflet.
			\item Caricamento mappa.
			\item Creazione delle icone, dei markers e dei popups.
		\end{enumerate}
	\bigskip 
	\item \LARGE{\textbf{SPARQL}}
		\begin{enumerate}
			\Large
			\item Introduzione.
			\item Il modello Turtle.
			\item Libreria javascript per interrogazioni SPARQL.
		\end{enumerate}
	\bigskip 
	\item \LARGE{\textbf{Ontologie per la rappresentazione dei servizi pubblici}}
		\begin{enumerate}
			\Large
			\item Menù gerarchici.
			\item Codifica JSON.
			\item Chiamata AJAX.
		\end{enumerate}
	\bigskip 
	\item \LARGE{\textbf{Presentazione e codice dell'applicazione}}
		\begin{enumerate}
			\Large
			\item Programmi utilizzati.
			\item HTML.
			\item Javascript.
			\item Css.
		\end{enumerate}
\end{enumerate}
\bigskip 
		
\textbf {Appendice} Il problema delle richieste Cross-Domain.
\newpage

\begin{enumerate}
	\item \LARGE{\textbf{Leaflet}}
		\begin{enumerate}
			\Large
			\item {Introduzione}\newline
La realizzazione del portale è stata resa possibile grazie alle librerie fornite dal sito web Leaflet, accessibile digitando l'indirizzo url http://leafletjs.com/.
			Leaflet è una moderna libreria open-source realizzata in JavaScript e ha lo scopo di rendere interattive le mappe per utilizzarle in qualsiasi piattaforma si voglia, che sia desktop o mobile. Lo sviluppatore di tale libreria è Vladimir Agafonkin che, con l'aiuto team di collaboratori dedicati, ha realizzato una semplice e versatile libreria con soli circa 33 KB di memoria, inoltre utilizzando la tecnologia \textbf{HTML5} e \textbf{CSS3} è accessibile sui browser moderni quali Chrome, Firefox, Safari e Internet Explorer. Può essere anche accessibile per i borwser più datati e ha anche una buona e facile documentazione on-line, infine ha un'estesa e vasta gamma di plugin, che si possono facilmente integrare rendendo il più compatto possibile e di facile intuzione.	
			\medskip
			\item Collegamento con la libreria\newline
Per preparare il sito web con la mappa interattiva occorre realizzare le seguenti principali procedure:
			\begin{itemize}
				\item Inserire nel codice HTML nella sezione \textbf{head} il riferimento al file \textbf{"`leaflet.css"`}
				\item Includere il file scritto in JavaScript \textbf{"`leaflet.js"'}
				\item Inserire un elemento div che ha come parametro \textbf{"`id=map"'} nella sezione \textbf{body}
				\item Settare attraverso la tecnologia CSS3 le caratteristiche della mappa attraverso l'id "`map"'.				
			\end{itemize}
Come mostrato in Figura 1 e in Figura 2

			\begin{figure}[!ht]
				
			\end{figure}
			\tab {\textbf{Figura 1}}
			\newpage
			\begin{figure}[!ht]
				
			\end{figure}
			\tab{\textbf{Figura 2}}
				\newline				
			\medskip
			
Il secondo passaggio è quello di creare una mappa interattiva, per farlo occorre collegarsi al sito \textbf{www.mapbox.com}, che fornisce un portale gratuito per creare o modificare una mappa secondo le caratteristiche che si vogliono.\newline
Per fare questo occorre registrarsi al sito web fornendo:
			\smallskip
			\begin{itemize}
				\item Username
				\item Cognome
				\item Nome
				\item Email
				\item Password
			\end{itemize}
			\smallskip
			\medskip
Una volta effettuato l'accesso bisogna andare sulla sezione \textbf{Project} e cliccare sul pulsante \textbf{Create Project}, in questo modo creerà una nuova mappa da poter modellare a seconda delle proprie necessità.\newline
Sul pannello in alto a sinistra è indicata la consolle per modellare la mappa. Con il pannello \textbf{Style} possiamo scegliere i colori delle strade, dell'acqua, della terra e pianure e dello sfondo.\newline
Sul pannello \textbf{Data}, invece, possiamo decidere se aggiungere marker, poligoni o linee. Di default verrà utilizzato il colore blue.
Infine sull'ultimo pannello, chiamato \textbf{Project} sono indicate tutti i parametri che servono per caricare la mappa sul portale web, come ad esempio il campo Map ID.
Sistemate tutte le modifiche per confermare il progetto della mappa cliccare sul pulsante Save.\newline
In Figura 3 sono mostrate tutti i pannelli con i relativi campi.
			\newpage
			\begin{figure}[!ht]
				
			\end{figure}
			\tab{\textbf{Figura 3}}
				\newline				
			\medskip
			
Per inserire la mappa sul codice sorgente e caricarla nella pagina web, bisogna inizializzarla fornendo le coordinate geografiche della posizione e il livello dello zoom.
Tali informazioni sono reperibile sul sito \textbf{www.mapbox.com} in basso a sinistra come mostrato in Figura 4, e devono essere insirite nella sezione body della file \textbf{index.html}.
			\begin{figure}[!ht]
			
			\end{figure}
			
			\tab{\textbf{Figura 4}}
				\newline
	
Il codice che bisogna inserire è il seguente:
			\begin{figure}[!ht]
				
			\end{figure}
			
			\tab{\textbf{Figura 5}}
				\newline
				
Dalla Figura 5 si evince che all'interno delle parentesi, sono inserite le coordinate geografiche presenti nella Figura 4, inoltre lo script utilizzato è sempre JavaScript.
Il prossimo passaggio è quello di inserire la mappa utilizzando il comando \textbf{tileLayer}. Tale comando permette di inserire la mappa creata sul sito di Mapbox, definendo lo zoom massimo possibile, gli attributi e l'identificativo della mappa.\newline
Di seguito abbiamo il codice sorgente:
			\begin{figure}[!ht]
				
			\end{figure}
			
			\tab{\textbf{Figura 6}}
				\newline

Sulla Figura 6 è indicato il codice da digitare per il carimento della mappa, si nota che nella prima voce è indicato il collegamento ipertestuale della mappa, alla voce \textbf{attributes} sono indicate le licenze, mentre alla voce \textbf{id}, è specificato l'identificativo della nostra mappa, che si ricava dal sito Mapbox alla sezione \textbf{Project} come mostrato in Figura 3 alla voce Map ID.\newline
Seguendo le procedure appena indicate, la mappa verrà ricavata correttamente nella nostra pagina web.
		\medskip
		\item {Creazione dell'icona}\newline
Attraverso il metodo \textbf{L.Icon} è possibile creare un icona che identifica un determinato punto della mappa. Per prima cosa occorre scegliere un'immagine, dopodiché attraverso \textbf{iconUrl} è possibile caricarla specificando il percorso del file; se si dispone anche di un immagine con l'ombra bisogna caricarla con \textbf{shadowUrl} sempre specificando il percorso del file. Infine bisogna specificare la grandezza dell'icona e dell'ombra, attraverso i parametri \textbf{iconSize} e \textbf{shadowSize} come è evidenziato nella Figura 7.
			\newpage
			\begin{figure}[!ht]
				
			\end{figure}
			
			\tab{\textbf{Figura 7}}
			\newline

Fatto questo inseriamo l'icona sulla mappa specificando le coordinate geografiche e la variabile \textbf{iconBlue} attraverso il metodo \textbf{L.marker}

			\begin{figure}[!ht]
				
			\end{figure}
			
			\tab{\textbf{Figura 8}}
			\newline
		\end{enumerate}
	\newpage
	\item \LARGE{\textbf{Libreria SPARQL per javascript}}
		\begin{enumerate}
			\Large
			 
			\item {Introduzione}\newline
			\item {Collegamento con la libreria}\newline
			\item {Implementazione}\newline
		
		\end{enumerate}
\end{enumerate}
\end{document}
